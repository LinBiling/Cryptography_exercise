\documentclass[paper=a4, fontsize=11pt]{scrartcl} % A4 paper and 11pt font size


 \usepackage[T1]{fontenc} % Use 8-bit encoding that has 256 glyphs
 \usepackage{fourier} % Use the Adobe Utopia font for the document - comment this line to return to the LaTeX default
 \usepackage[english]{babel} % English language/hyphenation
 \usepackage{amsmath,amsfonts,amsthm} % Math packages
 \usepackage{graphicx}
 \usepackage{lipsum} % Used for inserting dummy 'Lorem ipsum' text into the template

 \usepackage{sectsty} % Allows customizing section commands
 \allsectionsfont{\centering \normalfont\scshape} % Make all sections centered, the default font and small caps


 \usepackage{fancyhdr} % Custom headers and footers
 \pagestyle{fancyplain} % Makes all pages in the document conform to the custom headers and footers
 \fancyhead{} % No page header - if you want one, create it in the same way as the footers below
 \fancyfoot[L]{} % Empty left footer
 \fancyfoot[C]{} % Empty center footer
 \fancyfoot[R]{\thepage} % Page numbering for right footer
 \renewcommand{\headrulewidth}{0pt} % Remove header underlines
 \renewcommand{\footrulewidth}{0pt} % Remove footer underlines
 \setlength{\headheight}{13.6pt} % Customize the height of the header


 \numberwithin{equation}{section} % Number equations within sections (i.e. 1.1, 1.2, 2.1, 2.2 instead of 1, 2, 3, 4)
 \numberwithin{figure}{section} % Number figures within sections (i.e. 1.1, 1.2, 2.1, 2.2 instead of 1, 2, 3, 4)
 \numberwithin{table}{section} % Number tables within sections (i.e. 1.1, 1.2, 2.1, 2.2 instead of 1, 2, 3, 4)


 \setlength\parindent{0pt} % Removes all indentation from paragraphs - comment this line for an assignment with lots of text


 %----------------------------------------------------------------------------------------
 %       TITLE SECTION
 %----------------------------------------------------------------------------------------


 \newcommand{\horrule}[1]{\rule{\linewidth}{#1}} % Create horizontal rule command with 1 argument of height


 \title{
 \normalfont \normalsize
 \textsc{Jiangsu University, school of Computer Science and Communication Engineering} \\ [25pt] % Your university, school and/or department name(s)
 \horrule{0.5pt} \\[0.4cm] % Thin top horizontal rule
 \huge Exercises \\ % The assignment title
 \horrule{2pt} \\[0.5cm] % Thick bottom horizontal rule
 }


 \author{Xia Wentao} % Your name


 \date{\normalsize\today} % Today's date or a custom date


 \begin{document}


 \maketitle % Print the title
 \section{Classic cryptography}
 \label{sec:cc}

\textbf{1.}
Briefly define the Caesar cipher.\\

\textbf{Answer:}\\
The Caesar cipher involves replacing each letter of the alphabet with the letter standing k places further down the alphabet, for k in the range 1 through 25.\\
\\
\textbf{2.}
Briefly define the monoalphabetic cipher.\\

\textbf{Answer:}\\
A monoalphabetic substitution cipher maps a plaintext alphabet to a ciphertext alphabet, so that each letter of the plaintext alphabet maps to a single unique letter of the ciphertext alphabet.\\

\textbf{3.}
Briefly define the Playfair cipher.\\

\textbf{Answer:}\\
The Playfair algorithm is based on the use of a $5 \times 5$ matrix of letters constructed using a keyword. Plaintext is encrypted two letters at a time using this matrix.\\

\textbf{4.}
What is the difference between a monoalphabetic cipher and a polyalphabetic cipher?\\

\textbf{Answer:}\\
A polyalphabetic substitution cipher uses a separate monoalphabetic substitution cipher for each successive letter of plaintext, depending on a key.\\

\textbf{5.}
What is the difference between a monoalphabetic cipher and a polyalphabetic cipher?\\

\textbf{Answer:}\\
A polyalphabetic substitution cipher uses a separate monoalphabetic substitution cipher for each successive letter of plaintext, depending on a key.\\

\textbf{6.}
What are two problems with the one-time pad?\\

\textbf{Answer:}\\
1. There is the practical problem of making large quantities of random keys. Any heavily used system might require millions of random characters on a regular basis. Supplying truly random characters in this volume is a significant task.\\
2. Even more daunting is the problem of key distribution and protection. For every message to be sent, a key of equal length is needed by both sender and receiver. Thus, a mammoth key distribution problem exists.\\

\textbf{7.}
What are two problems with the one-time pad?\\

\textbf{Answer:}\\
1. There is the practical problem of making large quantities of random keys. Any heavily used system might require millions of random characters on a regular basis. Supplying truly random characters in this volume is a significant task.\\
2. Even more daunting is the problem of key distribution and protection. For every message to be sent, a key of equal length is needed by both sender and receiver. Thus, a mammoth key distribution problem exists.\\

\textbf{8.}
What is a transposition cipher?\\

\textbf{Answer:}\\
A transposition cipher involves a permutation of the plaintext letters.\\

\textbf{9.}
What is steganography?\\

\textbf{Answer:}\\
Steganography involves concealing the existence of a message.\\

\textbf{10.}
A ciphertext has been generated with an affine cipher.The most frequent letter of the
ciphertext is $'B'$, and the second most frequent letter of the ciphertext is $'U'$. Break
this code.\\

\textbf{Answer:}\\
Assume that the most frequent plaintext letter is e and the second most frequent letter is t. Note that the numerical values are e = 4; B = 1; t = 19; U = 20.\\ Then we have the following equations:

    $1 = (4a + b)$ mod $26$
	20 = (19a + b) mod 26

	Thus, 19 = 15a mod 26. \\
    By trial and error, we solve: a = 3.\\
    Then 1 = (12 + b) mod 26. By observation, b = 15.\\
    
\textbf{11.}
One way to solve the key distribution problem is to use a line from a book that both
the sender and the receiver possess. Typically, at least in spy novels, the first sentence
of a book serves as the key. The particular scheme discussed in this problem is from
one of the best suspense novels involving secret codes, $Talking$  $to$  $Strange$  $Men$ $,$  $by$
 $Ruth$  $Rendell$. Work this problem without consulting that book!
Consider the following message:\\
\textbf{SIDKHKDM AF HCRKIABIE SHIMC KD LFEAILA}\\
This ciphertext was produced using the first sentence of $The$ $Other$ $Side$ $of$ $Silence$
(a book about the spy Kim Philby):\\
The snow lay thick on the steps and the snowflakes driven by the wind
looked black in the headlights of the cars.
A simple substitution cipher was used.\\
a. What is the encryption algorithm? \\
b. How secure is it?\\
c. To make the key distribution problem simple, both parties can agree to use the
first or last sentence of a book as the key. To change the key, they simply need to
agree on a new book. The use of the first sentence would be preferable to the use
of the last. Why?\\

\textbf{Answer:}\\
a.	The first letter t corresponds to A, the second letter h corresponds to B, e is C, s is D, and so on. Second and subsequent occurrences of a letter in the key sentence are ignored. The result\\

{ ciphertext:  SIDKHKDM AF HCRKIABIE SHIMC KD LFEAILA}\\
{ plaintext:   basilisk  to  leviathan  blake  is  contact}\\

	b.	It is a monalphabetic cipher and so easily breakable.
	c.	The last sentence may not contain all the letters of the alphabet. If the first sentence is used, the second and subsequent sentences may also be used until all 26 letters are encountered.\\

\textbf{12.}
In one of his cases, Sherlock Holmes was confronted with the following message.\\
$534$ $C2$ $13$ $127$ $36$ $31$ $4$ $17$ $21$ $41$\\
DOUGLAS 109 293 5 37 BIRLSTONE\\
26 BIRLSTONE 9 127 171\\
Although Watson was puzzled, Holmes was able immediately to deduce the type of
cipher. Can you?\\

\textbf{Answer:}\\
The cipher refers to the words in the page of a book. The first entry, 534, refers to page 534. The second entry, C2, refers to column two. The remaining numbers are words in that column. The names DOUGLAS and BIRLSTONE are simply words that do not appear on that page. Elementary! (from The Valley of Fear, by Sir Arthur Conan Doyle)\\

\textbf{13.}
When the PT-109 American patrol boat, under the command of Lieutenant John F.
Kennedy, was sunk by a Japanese destroyer, a message was received at an Australian
wireless station in Playfair code:\\
\begin{center}
KXJEY UREBE ZWEHE WRYTU HEYFS\\
KREHE GOYFI WTTTU OLKSY CAJPO\\
BOTEI ZONTX BYBNT GONEY CUZWR\\
GDSON SXBOU YWRHE BAAHY USEDQ\\
\end{center}
The key used was royal new zealand navy. Decrypt the message. Translate TT into tt.\\

\textbf{Answer:}\\
PT BOAT ONE OWE NINE LOST IN ACTION IN BLACKETT STRAIT TWO MILES SW MERESU COVE X CREW OF TWELVE X REQUEST ANY INFORMATION\\

\textbf{14.}
a. How many possible keys does the Playfair cipher have? Ignore the fact that some
keys might produce identical encryption results. Express your answer as an
approximate power of 2.\\
b. Now take into account the fact that some Playfair keys produce the same encryption results. How many effectively unique keys does the Playfair cipher have?\\

\textbf{Answer:}\\
a.	$25!\approx 2^{84}$\\
	b.	Given any 5x5 configuration, any of the four row rotations is equivalent, for a total of five equivalent configurations. For each of these five configurations, any of the four column rotations is equivalent. So each configuration in fact represents 25 equivalent configurations. Thus, the total number of unique keys is 25!/25 = 24!\\

\textbf{15.}
In one of Dorothy Sayers��s mysteries, Lord Peter is confronted with the message
shown in Figure 2.10. He also discovers the key to the message, which is a sequence of
integers:
787656543432112343456567878878765654
3432112343456567878878765654433211234
a. Decrypt the message. Hint: What is the largest integer value?
b. If the algorithm is known but not the key, how secure is the scheme?
c. If the key is known but not the algorithm, how secure is the scheme?\\

\textbf{Answer:}\\
a.	Lay the message out in a matrix 8 letters across. Each integer in the key tells you which letter to choose in the corresponding row. Result:

	He sitteth between the cherubims. The isles may be glad thereof. As the rivers in the south.

	b.	Quite secure. In each row there is one of eight possibilities. So if the ciphertext is 8n letters in length, then the number of possible plaintexts is 8n.
	c.	Not very secure. Lord Peter figured it out. (from The Nine Tailors)\\

\textbf{16.}
A generalization of the Caesar cipher, known as the affine Caesar cipher, has the following form: For each plaintext letter p , substitute the ciphertext letter C:

\centerline{C = E([a, b], p) = (ap + b) mod 26}

A basic requirement of any encryption algorithm is that it be one-to-one. That is, if p$\not\equiv$q,then E(k,p)$\not\equiv$E(k,q) . Otherwise, decryption is impossible, because more than one plaintext character maps into the same ciphertext character. The affine Caesar cipher is not one-to-one for all values of a. For example, for a=2 and b=3,then E([a,b],0)=E([a,b],13)=3.

a. Are there any limitations on the value of b Explain why or why not.

b. Determine which values of are not allowed.

c. Provide a general statement of which values of a are and are not allowed. Justify your statement.\\

\textbf{Answer:}\\
a. No. A change in the value of b shifts the relationship between plaintext letters and ciphertext letters to the left or right uniformly, so that if the mapping is one-to-one it remains one-to-one.

b. 2,4,6,8,10,12,13,14,16,18,20,22,24. Any value of a larger than 25 is equivalent to a mod 26.

c. The values of a and 26 must have no common positive integer factor other than 1. This is equivalent to saying that a and 26 are relatively prime, or that the greatest common divisor of a and 26 is 1. To see this, first note that E(a,p) = E(a,q) (0$\le$p$\le$q<26) if and only if a(p-q) is divisible by 26.

(1). Suppose that a and 26 are relatively prime. Then, a(p-q) is not divisible by 26, because there is no way to reduce the fraction a/26 and (p-q) is less than 26.

(2). Suppose that a and 26 have a common factor k>1. Then E(a,p)=E(a,q), if q=p+m/k$\not\equiv$p.\\

\textbf{17.}
A ciphertext has been generated with an affine cipher.The most frequent letter of the ciphertext is B, and the second most frequent letter of the ciphertext is U. Break this code.\\

\textbf{Answer:}\\Assume that the most frequent plaintext letter is e and the second most frequent letter is t. Note that the numerical values are e = 4; B = 1; t = 19; U = 20. Then we have the following equations:\\
	\centerline{1 = (4a + b) mod 26}\\
	\centerline{20 = (19a + b) mod 26}\\
	Thus, 19 = 15a mod 26. By trial and error, we solve: a = 3.
	Then 1 = (12 + b) mod 26. By observation, b = 15.\\

\textbf{18.}
Using the Vigenere cipher, encrypt the word $"$explanation$"$ using the key leg.\\
\\
\textbf{Answer:}\\key:	legleglegle\\
    plaintext:	explanation\\
	ciphertext:	PBVWETLXOZR\\

\textbf{19.}
Consider a Feistel cipher composed of sixteen rounds with a block length of 128 bits and a key length of 128 bits. Suppose that, for a given k,the key scheduling algorithm determines values for the first eight round keys,k1,k2 and then sets:\\
\centerline{k9=k8,k10=k7,k11=k6.....k16=k1}\\
Suppose you have a ciphertext c. Explain how, with access to an encryption oracle,you can decrypt c and determine m using just a single oracle query. This shows that such a cipher is vulnerable to a chosen plaintext attack. (An encryption oracle can be thought of as a device that, when given a plaintext, returns the corresponding ciphertext. The internal details of the device are not known to you and you cannot break open the device. You can only gain information from the oracle by making queries to it and observing its responses.)\\
\\
\textbf{Answer:}\\Because of the key schedule, the round functions used in rounds 9 through 16 are mirror images of the round functions used in rounds 1 through 8. From this fact we see that encryption and decryption are identical. We are given a ciphertext c. Let m' = c. Ask the encryption oracle to encrypt m'. The ciphertext returned by the oracle will be the decryption of c.\\

\textbf{20.}
It can be shown that the Hill cipher with the matrix
\begin{equation}       %��ʼ��ѧ����
\left(                 %������
  \begin{array}{cc}   %�þ���һ��2�У�ÿһ�ж����з���
    a & b\\  %��һ��Ԫ��
    c & d\\  %�ڶ���Ԫ��
  \end{array}
\right)                 %������
\end{equation}
requires that is relatively prime to 26; that is, the only common positive integer factor of and 26 is 1. Thus, if or is even, the matrix is not allowed. Determine the number of different (good) keys there are for a 2$\times$ 2 Hill cipher without counting them one by one, using the following steps:\\
a. Find the number of matrices whose determinant is even because one or both rows are even. (A row is $"$even$"$ if both entries in the row are even.)\\
b. Find the number of matrices whose determinant is even because one or both columns are even. (A column is $"$even$"$ if both entries in the column are even.)\\
c. Find the number of matrices whose determinant is even because all of the entries are odd.\\
d. Taking into account overlaps, find the total number of matrices whose determinant is even.\\
e. Find the number of matrices whose determinant is a multiple of 13 because the first column is a multiple of 13.\\
f. Find the number of matrices whose determinant is a multiple of 13 where the first column is not a multiple of 13 but the second column is a multiple of the first modulo 13.\\
g. Find the total number of matrices whose determinant is a multiple of 13.\\
h. Find the number of matrices whose determinant is a multiple of 26 because they fit cases parts (a) and (e), (b) and (e), (c) and (e), (a) and (f), and so on.\\
i. Find the total number of matrices whose determinant is neither a multiple of 2 nor a multiple of 13.\\
\\
\textbf{Answer:}\\
    a.7$\times$ $13^{4}$\\
	b.7$\times$ $13^{4}$\\
	c.$13^{4}$\\
	d.10$\times$ $13^{4}$\\
	e.24$\times$ $13^{2}$\\
	f.24$\times$ ($13^{2}$-1)\\
	g.37648\\
	h.23530\\
	i.157248\\


\end{document}


