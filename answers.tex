\documentclass[paper=a4, fontsize=11pt]{scrartcl} % A4 paper and 11pt font size

\usepackage[T1]{fontenc} % Use 8-bit encoding that has 256 glyphs
\usepackage{fourier} % Use the Adobe Utopia font for the document - comment this line to return to the LaTeX default
\usepackage[english]{babel} % English language/hyphenation
\usepackage{amsmath,amsfonts,amsthm} % Math packages

\usepackage{lipsum} % Used for inserting dummy 'Lorem ipsum' text into the template

\usepackage{sectsty} % Allows customizing section commands
\allsectionsfont{\centering \normalfont\scshape} % Make all sections centered, the default font and small caps

\usepackage{fancyhdr} % Custom headers and footers
\pagestyle{fancyplain} % Makes all pages in the document conform to the custom headers and footers
\fancyhead{} % No page header - if you want one, create it in the same way as the footers below
\fancyfoot[L]{} % Empty left footer
\fancyfoot[C]{} % Empty center footer
\fancyfoot[R]{\thepage} % Page numbering for right footer
\renewcommand{\headrulewidth}{0pt} % Remove header underlines
\renewcommand{\footrulewidth}{0pt} % Remove footer underlines
\setlength{\headheight}{13.6pt} % Customize the height of the header

\numberwithin{equation}{section} % Number equations within sections (i.e. 1.1, 1.2, 2.1, 2.2 instead of 1, 2, 3, 4)
\numberwithin{figure}{section} % Number figures within sections (i.e. 1.1, 1.2, 2.1, 2.2 instead of 1, 2, 3, 4)
\numberwithin{table}{section} % Number tables within sections (i.e. 1.1, 1.2, 2.1, 2.2 instead of 1, 2, 3, 4)

\setlength\parindent{0pt} % Removes all indentation from paragraphs - comment this line for an assignment with lots of text

%----------------------------------------------------------------------------------------
%       TITLE SECTION
%----------------------------------------------------------------------------------------

\newcommand{\horrule}[1]{\rule{\linewidth}{#1}} % Create horizontal rule command with 1 argument of height

\title{
\normalfont \normalsize
\textsc{Jiangsu University, school of Computer Science and Communication Engineering} \\ [25pt] % Your university, school and/or department name(s)
\horrule{0.5pt} \\[0.4cm] % Thin top horizontal rule
\huge Exercises \\ % The assignment title
\horrule{2pt} \\[0.5cm] % Thick bottom horizontal rule
}

\author{Zhendong Yang} % Your name

\date{\normalsize\today} % Today's date or a custom date

\begin{document}

\maketitle % Print the title
\section{Symmetric key cryptosystem}
\label{sec:skc}
everyone should writes a distinct section provided with 20 excercise.
\section{Public key cryptosystem}
\label{sec:pkc}

\section{Access control}

\section{Classic cryptography}
\label{sec:cc}

\section{Digital signature}
\label{sec:ds}

\section{Confidentiality and integrity policies}

\section{Key Management}
\label{sec:km}
\subsection{Diffie-Hellman Key Exchange Problem \uppercase\expandafter{\romannumeral1}}

Users A and B use the Diffie-Hellman key exchange technique with a common prime $q = 71$ and a primitive root $a = 7$.

Question: 
\begin{enumerate}
\item If user A has private key, what is A's public key $Y_A$?
\item If user B has private key, what is B's public key $Y_B$?
\item  What is the shared secret key?
\end{enumerate}

Answer: 
\begin{enumerate}
\item $Y_A = 7^5 mod 71= 51$.
\item $Y_B = 7^{12} mod 71= 4$.
\item $K = 4^5 mod 71= 30$.
\end{enumerate}


\subsection{Diffie-Hellman Key Exchange Problem \uppercase\expandafter{\romannumeral2}}

Consider a Diffie-Hellman scheme with a common prime $q = 11$ and a primitive root $\alpha = 7$.

Question: 
\begin{enumerate}
\item Show that 2 is a primitive root of 11.
\item If user A has public key $Y_A = 9$, what is A��s private key $X_A$?
\item If user B has public key $Y_B = 3$, what is the secret key $K$ shared with $A$?
\end{enumerate}


Answer: 
\begin{enumerate}
\item $\phi(11) = 10$, $2^10 = 1024 = 1 mod 11$. If you check $2^n$ for $n < 10$, you will find that none of the values is $1 mod 11$.
\item 6, because $2^6 mod 11 = 9$.
\item $K = 3^6 mod 11= 3$.
\end{enumerate}


\subsection{Diffie-Hellman Protocol Problem \uppercase\expandafter{\romannumeral1}}

In the Diffie-Hellman protocol, each participant selects a secret number $x$ and sends the other participant $\alpha^x mod q$ for some public number $\alpha$. What would happen if the participants sent each other $x^{\alpha}$ for some public number $\alpha$ instead? Give at least one method Alice and Bob could use to agree on a key. Can Eve break your system without finding the secret numbers? Can Eve find the secret numbers?

Answer:

For example, the key could be $x_A^gx_B^g = {\left( {{x_A}{x_B}} \right)^g}$. Of course, Eve can find that trivially just by multiplying the public information. In fact, no such system could be secure anyway, because Eve can find the secret numbers $x_A$ and $x_B$ by using Fermat��s Little Theorem to take g-th roots.

\subsection{Diffie-Hellman Protocol Problem \uppercase\expandafter{\romannumeral2}}

This problem illustrates the point that the Diffie-Hellman protocol is not secure without the step where you take the modulus; i.e. the "Indiscrete Log Problem" is not a hard problem! You are Eve and have captured Alice and Bob and imprisoned them. You overhear the following dialog.

\begin{enumerate}
\item\textbf{Bob}: Oh, let's not bother with the prime in the Diffie-Hellman protocol, it will make things easier.
\item\textbf{Alice}: Okay, but we still need a base $\alpha$ to raise things to. How about $g = 3$?
\item\textbf{Bob}: All right, then my result is 27.
\item\textbf{Alice}: And mine is 243.
\end{enumerate}

What is Bob's secret and Alice's secret? What is their secret combined key? (Don't forget to show your work.)

Answer:
\begin{enumerate}
\item $x_B = 3$.
\item $x_A = 5$.
\item the secret combined key is $(3^3)^5 = 3^{15} = 14348907$.
\end{enumerate}

\subsection{Diffie-Hellman Key Exchange Problem \uppercase\expandafter{\romannumeral2}}

We describes a man-in-the-middle attack on the Diffie-Hellman key exchange protocol in which the adversary generates two public�Cprivate key pairs for
the attack. Could the same attack be accomplished with one pair? Explain.

Answer: 
\begin{enumerate}
\item Darth prepares for the attack by generating a random private key $X_D$ and then computing the corresponding public key $Y_D$.
\item Alice transmits $Y_A$ to Bob.
\item Darth intercepts $Y_A$ and transmits $Y_D$ to Bob. Darth also calculates $K2 = {\left( {{Y_A}} \right)^{{X_D}}}\bmod q$.
\item Bob receives $Y_D$ and calculates $K1 = {\left( {{Y_D}} \right)^{{X_B}}}\bmod q$.
\item Bob transmits $Y_A$ to Alice.
\item Darth intercepts $Y_A$ and transmits $Y_D$ to Alice. Darth calculates $K1 = {\left( {{Y_B}} \right)^{{X_D}}}\bmod q$.
\item Alice receives $Y_D$ and calculates $K2 = {\left( {{Y_D}} \right)^{{X_A}}}\bmod q$.
\end{enumerate}

















\section{virus}
\label{sec:vs}


\section{Cryptographic Checksums}
\label{sec:checksums}


\end{document}
